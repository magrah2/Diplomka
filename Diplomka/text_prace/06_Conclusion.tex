\chapter*{Conclusion}
\addcontentsline{toc}{chapter}{Conclusion}

In this thesis our main goal was to study the motion of classical relativistic strings in a~curved spacetime, specifically in a~constantly expanding universe and in flat spacetime with gravitational wave background. Throughout the way, we also derived very useful tools for solving the equations of motion of a~classical string in curved spacetime.

In the first chapter, we explained the difference between solving equations of motion for a~particle and solving them for a~string. We have intuitively shown how the action looks like and derived the equations of motion together with the boundary conditions. 

In the second chapter, we briefly studied a~classical string in flat spacetime. We have demonstrated how the choice of parameterization can simplify the equations of motion. Also, we have shown how to calculate the rest energy of a~string, which is used in the fourth chapter.

In the third chapter we took a~look at circular strings in a~constantly expanding universe. We have briefly described why the metric of an expanding universe has this form. Moreover, we used the action principle developed in the first chapter and derived the equations of motion. Because of the complicated nature of these equations, we resorted to different approach such as calculating a~potential and using the conservation of energy. This allowed us to find all solutions and describe how the string is affected by this expansion of the universe. We found, that the expansion of the universe has a~very small effect on circulars strings with small radii, but can be devastating for large strings with the rest energy greater than some critical energy. In that case, the string expand beyond all limits or until they break. 

In the final, fourth chapter, we studied relativistic strings in flat spacetime, but with gravitational wave background. First, we have shown that the used form of the metric is a~solution to the vacuum Einstein equations and what it implies on its components. Second, with the right choice of parameterization, we arrived at equations of motion on periodic gravitational wave background. Furthermore, we have rewritten two of these equations into the form of the Mathieu equation and discussed the regions of stability and instability of the solutions. Last but not least, we looked at how the string is affected by a~gravitational wave burst. This allowed us to properly evaluate the effect of this burst on an~initially circular string. We found, that the effect of gravitational waves on strings is amplified, when the frequency of the string and the gravitational wave are in or close to resonance. The string then receives large amounts of energy and momentum in the direction of propagation of the gravitational wave.

Overall this thesis achieved the goals of finding solutions for strings in various curved spacetimes. It also allowed us to learn about the technical and mathematical difficulties of general relativity and string theory and how to overcome them.


