\chapter*{Introduction}
\addcontentsline{toc}{chapter}{Introduction}

In this thesis, we will study the motion of classical relativistic strings on various backgrounds. The superstring theory is an attempt to explain all of the particles and fundamental forces in nature in one theory. It is a very complicated theory that requires plenty of mathematical prowess. The superstring theory pursues the strings as quantized, which complicates the calculations. We chose to study classical strings which still provide a good insight and intuition into the matters of the string theory and many mathematically interesting topics.

In the first chapter, we will show the differences between the action principle for particles and for strings in the general relativity. Furthermore, we will find a~convenient form of the equations of motion and the boundary conditions, that must be imposed on the string's endpoints.

In the second chapter, we will briefly look at the behavior of strings in flat spacetime. We will also show, how the choice of parameterization of such a~string can help in solving its equations of motion.

In the third chapter, we will study a~circular string in a~constantly expanding universe. First, we will find a~potential for static strings, which indicates that there are some critical points for which the string has a~very interesting behavior. The equations of motion and the conservation of energy will help us find all trajectories in phase space and compare them to trajectories in flat spacetime.

The fourth chapter will be concerning strings on the plane gravitational wave background. Firstly, we will find a~suitable parameterization in which we can solve the equations of motion. Then we will choose two types of gravitational waves, namely periodic gravitational waves and a burst of gravitational waves with Gaussian envelope. In the former case we cannot interpret the results as easily as in the later case, where we have flat space-time before and after the gravitational wave burst. In the former case, we will only look at the stability of solutions, which will hint towards some resonance of the string with the~gravitational wave for specific frequencies. In the later case, we will compare the motion of the string before and after the gravitational wave burst.


